\chapter{Conclusions} \label{Conclusion}

The main goal of our thesis was to explore the capabilities of \emph{non-volatile memory} in the context of distributed systems. We did so by designing and implementing a system called \DHTS.
\DHTS is an NVM-enabled, highly available and scalable distributed hash table with tunable level of data replication. The system was designed following the principles of Amazon Dynamo and uses a custom-built generic local hash table implementation built on top of the Intel's \libpmemobj library.

Since we did not have access to persistent memory hardware, we emulated it in RAM. This, in turn, allowed us to measure the inherent overhead introduced by the additional computation performed by \libpmemobj in order to reliably store data in NVM. \PHT, our local NVM-enabled hash table that functions as the backend for \DHTS, was slower 2--10 times compared to the traditional hash table implemented in an analogous way but which relied solely on RAM. The observed difference in performance is significant, but still acceptable, given that the data kept in NVM is persistent. 

The results of the experimental evaluation of \DHTS indicate that it scales well with the increasing number of nodes.

NVM is a very promising new technology but it is not yet commonly recognised. Prior to the project start we knew very little about NVM, how it operates and how a programmer can harness its capabilities. The work on the thesis allowed us to learn how to design, implement and test applications that rely on it. We also gained valuable knowledge about the current limitations of the tools that we relied on, e.g., not very efficient memory management and the lack of the support for polymorphism in \libpmemobjcpp and not documented problems with application deployment caused by conflicts between the used libraries.

We also learned a great deal about the design of modern, highly-available services running on the Internet. \DHTS, similarly to such systems, lacks a central coordinator and is easily scalable, albeit at the cost of having only a simple \insertMethod/\getMethod/\removeMethod-method-based interface.

\section{Future work}

There are several ways in which our work can be further improved.

Firstly, we would like to implement a fully-fledged catch-up mechanism that ensures that the nodes that replicate the same data always eventually converge to the same state (the current implementation relies on reliable broadcast).

Secondly, we would like to extend \DHTS with a mechanism that allows one to easily iterate over the whole data set kept by the system's nodes. Currently we support only iteration over the data set on each node independently.

Finally, we would also like to test our system using hardware persistent memory and find out how it performs. Tests conducted in such an environment would give us a glimpse at the current performance and usability of the NVM technology. 

\endinput