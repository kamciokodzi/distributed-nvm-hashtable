
\chapter{Own work}

\section{NVM-enabled hashtable}

\subsection{Tests}

    To ensure that our hashtable works correctly, we developed a range of unit tests. As previously stated, we used the GoogleTest library. We decided to split our tests into two categories: logic and perfomance.
    
    Logic tests try to verify the correctness of basic operands used in our hashtable: insert, get, remove and iterate. We run them on two instances of hashtable: integer and string type. In first test we add a number of elements with known sum. Then we start iterating through hashmap, at the same time summing elements. At the end with assert we check for equality of those two sums. Next tests concentrates on insert, get and remove functionalities. We decided to verify them both in single such as multithread mode (using 8 threads), where one thread inserts 100 thousand elements. After completing the addition, in one test case we try to get previously added elements, while in the second one we try to remove them. For each element we compare received value with added one using an assert.
    
    In performance tests we concentrated on comparing the effectiveness of our hashtable with \textit{unordered\_map} from std namespace. We measured time for 8 threads to insert, get and remove elements. We conducted the same calculations for respectively 16, 8, 4, 2 and 1 thread.
    
    
    

        
    

% Rozdziały dokumentujące pracę własną studenta: opisujące ideę, sposób lub metodę 
% rozwiązania postawionego problemu oraz rozdziały opisujące techniczną stronę rozwiązania 
% --- dokumentacja techniczna, przeprowadzone testy, badania i uzyskane wyniki. 

% Praca musi zawierać elementy pracy własnej autora adekwatne do jego wiedzy praktycznej uzyskanej w
% okresie studiów. Za pracę własną autora można uznać np.: stworzenie aplikacji informatycznej lub jej
% fragmentu, zaproponowanie algorytmu rozwiązania problemu szczegółowego, przedstawienie projektu 
% np.~systemu informatycznego lub sieci komputerowej, analizę i ocenę nowych technologii lub rozwiązań
% informatycznych wykorzystywanych w przedsiębiorstwach, itp. 

% Autor powinien zadbać o właściwą dokumentację pracy własnej obejmującą specyfikację założeń i 
% sposób realizacji poszczególnych zadań
% wraz z ich oceną i opisem napotkanych problemów. W przypadku prac o charakterze 
% projektowo-implementacyjnym, ta część pracy jest zastępowana dokumentacją techniczną i użytkową systemu. 

% W pracy \textbf{nie należy zamieszczać całego kodu źródłowego} opracowanych programów. Kod źródłowy napisanych
% programów, wszelkie oprogramowanie wytworzone i wykorzystane w pracy, wyniki przeprowadzonych
% eksperymentów powinny być umieszczone na płycie CD, stanowiącej dodatek do pracy.

% \section*{Styl tekstu}

% Należy\footnote{Uwagi o stylu pochodzą częściowo ze stron Macieja Drozdowskiego~\cite{mdro}.} 
% stosować formę bezosobową, tj.~\emph{w pracy rozważono ......, 
% w ramach pracy zaprojektowano ....}, a nie: \emph{w pracy rozważyłem, w ramach pracy zaprojektowałem}. 
% Odwołania do wcześniejszych fragmentów tekstu powinny mieć następującą postać: ,,Jak wspomniano wcześniej, ....'', 
% ,,Jak wykazano powyżej ....''. Należy unikać długich zdań. 

% ,,Ilość'' i ,,liczba''. Proszę zauważyć, liczba dotyczy rzeczy policzalnych, np.~liczba osób, liczba zadań, procesorów. 
% Ilość dotyczy rzeczy niepoliczalnych, np.~ilość wody, energii. Należy starać się wyrażać precyzyjnie, tj.~zgodnie 
% z naturą liczonych obiektów.\footnote{(DW) Według wytycznych Rady Języka Polskiego obie formy są dopuszczalne
% zarówno do obiektów policzalnych, jak i niepoliczalnych. W tekstach technicznych warto być jednak precyzyjnym.}

% Niedopuszczalne są zwroty używane w języku potocznym. W pracy należy używać terminologii informatycznej, która ma 
% sprecyzowaną treść i znaczenie. Nie należy używać ,,gazetowych'' określeń typu: 
% silnik bazy danych, silnik programu, maszyna skryptowa, elektroniczny mechanizm, mapowanie, string, gdyż nie wiadomo 
% co one właściwie oznaczają. 

% Niedopuszczalne jest pisanie pracy metodą \emph{cut\&paste}, bo jest to plagiat i dowód intelektualnej indolencji autora.
% Dane zagadnienie należy opisać własnymi słowami. Zawsze trzeba powołać się na zewnętrzne źródła. 

