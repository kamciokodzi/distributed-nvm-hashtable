\chapter{Introduction}

Nowadays computers use two types of memories: Random Access Memory (RAM) for direct data access and Solid-State or Hard-Disk Drives (SSD, HDD) for its long-term storage.
Although RAM allows quick access to memory, the data it stores is volatile and therefore freed after each system shutdown.
On the other hand, both SSD's and HDD's provide persistent storage but imply a high cost of access.
To combine the best features of these two technologies, the \PM was developed. 
It is a type of computer memory which provides persistent data storage at the expense of slightly higher cost and lower performance than RAM.

The expensive access to storage problem is especially visible in peer-to-peer systems where the data is spread between nodes. 
If one of them crashes and needs to reboot, its data is lost. 
Reloading it each time from the disk space results in a significant time overhead.
Therefore, using non-volatile memory with direct access is a perfect choice for a distributed system.

We decided to get acquainted with the described technology by implementing a distributed hashmap which supports persistent memory. 
It should operate on a modern architecture with NVM on the processor's memory bus.
Despite the fact that the introduced solution is still in development, it can be emulated with RAM.

The main purpose of the following thesis is was design implement and test an \textbf{Nvm-enabled distributed hash table system}. 
To achieve this goal we used tools such as \textit{Persistent Memory Development Kit} \cite{PmemIo} and \textit{Boost} \cite{Asio}, mostly relying on their online documentation. 
We managed to implement a concurrent \textit{Persistent Hash Table} with support for the NVM technology. 
We developed a decentralised and symmetrical distributed system with support for the \textit{PHT}.
The created system is provides stable performance and scales linearly.

The structure of the this thesis goes as follows. 
The second chapter covers the tools and libraries used for implementation. 
It describes how NVM works, why does it require different programming approach and what are the several concepts of creating mentioned distributed system. 
It shortly describes used libraries, both used for the persistent memory support and distributing.
Chapter third is about Persistent Hash Table.
It states about certain design assumptions, discusses the implementation details and evaluates created solution in terms of performance.
Chapter four is dedicated to distribution of the mentioned structure.
It covers the system design and implementation, as well as used mechanisms such as consistent hashing or conflict resolutions.
At the end the system evaluation takes place.


%todo
Division of work

% Wstęp\footnote{Treść przykładowych rozdziałów została skopiowana
% z ,,zasad'' redakcji prac dyplomowych FCMu~\cite{fcm-red}.} do pracy powinien zawierać następujące elementy:
% \begin{itemize}
%     \item krótkie uzasadnienie podjęcia tematu; 
%     \item cel pracy (patrz niżej); 
%     \item zakres (przedmiotowy, podmiotowy, czasowy) wyjaśniający, w jakim rozmiarze praca będzie realizowana; 
%     \item ewentualne hipotezy, które autor zamierza sprawdzić lub udowodnić; 
%     \item krótką charakterystykę źródeł, zwłaszcza literaturowych; 
%     \item układ pracy (patrz niżej), czyli zwięzłą charakterystykę zawartości poszczególnych rozdziałów; 
%     \item ewentualne uwagi dotyczące realizacji tematu pracy np.~trudności, które pojawiły się w trakcie 
%     realizacji poszczególnych zadań, uwagi dotyczące wykorzystywanego sprzętu, współpraca z firmami zewnętrznymi. 
% \end{itemize}

% \noindent
% \textbf{Wstęp do pracy musi się kończyć dwoma następującymi akapitami:}
% \begin{quote}
% Celem pracy jest opracowanie / wykonanie analizy / zaprojektowanie / ...........
% \end{quote}
% oraz:
% \begin{quote}
% Struktura pracy jest następująca. W rozdziale 2 przedstawiono przegląd literatury na temat ........ 
% Rozdział 3 jest poświęcony ....... (kilka zdań). 
% Rozdział 4 zawiera ..... (kilka zdań) ............ itd. 
% Rozdział X stanowi podsumowanie pracy. 
% \end{quote}

% W przypadku prac inżynierskich zespołowych lub magisterskich 2-osobowych, po tych dwóch w/w akapitach 
% musi w pracy znaleźć się akapit, w którym będzie opisany udział w pracy poszczególnych członków zespołu. Na przykład:

% \begin{quote}
% Jan Kowalski w ramach niniejszej pracy wykonał projekt tego i tego, opracował ......
% Grzegorz Brzęczyszczykiewicz wykonał ......, itd. 
% \end{quote}

